%\documentclass[11pt,article,reqno]{amsart}
\documentclass[12pt,reqno]{article}
\usepackage[utf8]{inputenc}
\usepackage{enumerate}
\usepackage{lmodern}
\usepackage[T1]{fontenc}
\usepackage{verbatim}
\usepackage{textcomp}

\usepackage[english]{babel}
\usepackage[a4paper,vmargin={3.5cm,3.5cm},hmargin={2.5cm,2.5cm}]{geometry}
\usepackage[font=sf, labelfont={sf,bf}, margin=1cm]{caption}
\usepackage[pdftex]{hyperref}
\usepackage[pdftex]{color,graphicx}
\usepackage{xcolor}

% math packages:
\usepackage{amsmath,amsfonts,amssymb,amsthm,mathrsfs,mathtools,bbm}
\usepackage{empheq}
% Theorems
%-----------------------------------------------------------------
\newtheorem{theorem}{Theorem}[section]
\newtheorem{corollary}[theorem]{Corollary}
\newtheorem{lemma}[theorem]{Lemma}
\newtheorem{proposition}[theorem]{Proposition}
\newtheorem{fact}[theorem]{Fact}
\theoremstyle{remark}
\newtheorem{remark}[theorem]{Remark}
\theoremstyle{definition}
\newtheorem{definition}[theorem]{Definition}
\newtheorem{example}[theorem]{Example}
\newtheorem{assumption}[theorem]{Assumption}


% Shortcuts.
%-----------------------------------------------------------------

%Lazy shortcuts

\def\FNS{\mathbf{F}(\mathbf{N}_\sigma)}

% Blackboard
\def\RR{\mathbb{R}}
\def\R{\mathbb{R}}
\def\ZZ{\mathbb{Z}}
\def\NN{\mathbb{N}}
\def\QQ{\mathbb{Q}}
\def\PP{\mathbb{P}}
\def\XX{\mathbb{X}}
\def\YY{\mathbb{Y}}
\def\EE{\mathbb{E}}
\newcommand{\ex}{\mathbb{E}}
\def\Var{\mathbb{V}\mathrm{ar}}
\def\Cov{\mathbb{C}\mathrm{ov}}
\def\Corr{\mathbb{C}\mathrm{orr}}
\def\1{\mathbbm{1}}
% Greeks
\def\al{\alpha}
\def\be{\beta}
%\def\ga{\gamma}
%\def\de{\delta}
\def\om{\omega}
%\def\ep{\varepsilon}
\def\ph{\varphi}
\def\la{\lambda}
% Calligraphic
\newcommand{\cP}{\mathcal{P}}
\newcommand{\cX}{\mathcal{X}}
\newcommand{\cC}{\mathcal{C}}

\newcommand{\cY}{\mathcal{Y}}
\newcommand{\ytil}{\tilde{y}}

% Sans Serif
\def\sA{\mathsf{A}}
\def\sB{\mathsf{B}}
\def\sC{\mathsf{C}}
\def\sD{\mathsf{D}}
\def\sK{\mathsf{K}}
\def\sW{\mathsf{W}}
\def\sS{\mathsf{S}}
\def\sO{\mathsf{O}}
\def\sP{\mathsf{P}}
% Bold
\def\bx{\mathbf{x}}
\def\by{\mathbf{y}}
\def\bw{\mathbf{w}}
\def\bz{\mathbf{z}}
\def\bF{\mathbf{F}}
\def\b0{\mathbf{0}}
\def\bM{\mathbf{M}}
\def\bZ{\mathbf{Z}}
\def\bN{\mathbf{N}}
% Metrics
\def\dtv{d_{\mathrm{TV}}}
\def\dk{{d_\mathrm{K}}}
\def\dw{d_{\mathrm{W}}}
\def\dfm{d_{\mathrm{FM}}}
\def\d2{d_2}
\def\dc{d_\mathrm{c}}
\def\dh{d_\mathrm{H}}
% Decorations
%\def\ov{\overline}
\def\un{\underline}
\def\wt{\widetilde}
\def\wh{\widehat}
% Exotic Symbols
\def\con{\xleftrightarrow}
% Script
\def\scr{\mathscr}

\newcommand{\essinf}{{\rm ess~inf}}
\newcommand{\hyp}{{\rm Hyp}}
\newcommand{\bea}{\begin{eqnarray}}
\newcommand{\eea}{\end{eqnarray}}
\newcommand{\bean}{\begin{eqnarray*}}
\newcommand{\eean}{\end{eqnarray*}}
\newcommand{\eps}{\varepsilon}
\newcommand{\cN}{{\cal N}}
\newcommand{\Po}{{\cal P}}
\newcommand{\An}{A_n}%Alternative $A_{n,\alpha}$.
\renewcommand{\emptyset}{\varnothing}

\newcommand{\fmax}{f_{\rm max}}
\newcommand{\Cor}{\mathsf{Cor}}
\newcommand{\A}{\mathsf{A}}
\newcommand{\tod}{\overset{d}\longrightarrow}
\newcommand{\toP}{\overset{\PP}\longrightarrow}

\renewcommand{\eta}{\cP}
\newcommand{\Gum}{{\mathsf{Gu}}}
\newcommand{\PRV}{{\mathsf{Po}}}


\numberwithin{equation}{section}


% Notations of the article
% -----------------------------------------------------------------
%\usepackage[notref, notcite]{showkeys}
\DeclareMathOperator\NNG{NNG}
\DeclareMathOperator\MST{MST}
\DeclareMathOperator\diam{diam}
%\DeclareMathOperator\dist{dist}
\newcommand{\dist}{{\rm dist}}
\setcounter{footnote}{1}
% -----------------------------------------------------------------


\begin{document}
\title{\bf line coverage}
\author{}

\date{\today}

\maketitle

\begin{abstract}   
%\\
%\noindent{\bf Keywords:} \\
%\noindent{\bf AMS 2010 Classification:}
\end{abstract}
%\tableofcontents


 

\bibliographystyle{plain}
\begin{thebibliography}{123}

%% : 
%    single author :  Pen05 ...
%    several authors: BY05  ...
%

%\bibitem{AGG89} Arratia, R.; Goldstein, L.; Gordon, L. Two moments suffice for Poisson approximations: the Chen-Stein method. {\it Ann. Probab.} 17 (1989), no. 1, 9--25.
%
%\bibitem{BHJ} Barbour, A. D., Holst, L. and Janson, S. (1992)
%{\em Poisson Approximation.} Oxford University Press, Oxford.
%	Oxford Studies in Probability, 2.  Oxford Science Publications. {\it The Clarendon Press, Oxford University Press, New York}, 1992. {\rm x}+277 pp.
%
%\bibitem{Chen75} Chen, Louis H. Y. Poisson approximation for dependent trials. {\it Ann. Probab.} 3 (1975), no. 3, 534--545. 
%
%
%
%		
%\bibitem{HPY23}
%	Higgs, F., Penrose, M.D. and Yang, X. (2023)
%	Covering one point process with another.
%	Preprint, version from 4 September 2023.
%
%
%
%\bibitem{LP18} Last, G. %Günter 
%	and Penrose, M. (2018)  {\em Lectures on the Poisson process.}
%		%Institute of Mathematical Statistics Textbooks, 7.
%		 Cambridge University Press, Cambridge.
%		 %, 2018. xx+293 pp.
%
%\bibitem{Pen} Penrose, M. (2003) {\em Random Geometric Graphs}.
%	%Oxford Studies in Probability, 5. {\it
%	Oxford University Press, Oxford.
%	%}, 2003. xiv+330 pp.
%
%
%\bibitem{Pen18} Penrose, M. D. (2018) Inhomogeneous random graphs, 
%	isolated vertices, and Poisson approximation. 
%	{\it J. Appl. Probab.} {\bf 55},
%	%(2018), no. 1,
%	112--136.
%
%\bibitem{Pen22} Penrose, M. D. (2023) Random Euclidean coverage from within.
%	{\em Probab. Theory Related Fields} {\bf 185}, 747--814.


\end{thebibliography} 


\end{document}

